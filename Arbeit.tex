% Formatvorlage f�r studentische Arbeiten der Arbeitsgruppe Software Systems
% Engineering am Institut f�r Informatik der Universit�t Hildesheim
%   erstellt von Christopher Voges, 28.04.2016
%   �berarbeitet von Sascha El-Sharkawy, 06.08.2019
%   https://sse.uni-hildesheim.de/studium-lehre/richtlinien-fuer-ausarbeitungen/vorlagen/

% Dokumentenkopf 
\documentclass[
    11pt, % Schriftgr��e
    DIV=10,
    ngerman, % f�r Umlaute, Silbentrennung etc.
    a4paper, % Papierformat
    twoside, % weiseitiges Dokument
    titlepage, % es wird eine Titelseite verwendet
    parskip=half, % Abstand zwischen Abs�tzen (halbe Zeile)
    headings=normal, % Gr��e der �berschriften verkleinern
    listof=totoc, % Verzeichnisse im Inhaltsverzeichnis auff�hren
    bibliography=totoc, % Literaturverzeichnis im Inhaltsverzeichnis auff�hren
    index=totoc, % Index im Inhaltsverzeichnis auff�hren
    captions=tableheading, % Beschriftung von Tabellen unterhalb ausgeben
		numbers=noenddot,
    final, % Status des Dokuments (final/draft)
]{scrreprt}

% Einbinden der Packages
\input{Packages}

% Einbinden der Meta-Informationen
% Meta-Informationen 
% Falls Umlaute oder ein *�* vorkommen:
\usepackage[latin1]{inputenc}

% Hier k�nnen Sie Informationen zur Arbeit, sich selbst und Ihren Betreuern
% hinterlegen.
\newcommand{\titel}{Search-Based Test Data Generation} % Name der Arbeit
\newcommand{\untertitel}{} % Optional mit Untertitil
% Art der Arbeit ggf. zus�tzlich der Titel der Veranstaltung
\newcommand{\art}{Seminararbeit}
\newcommand{\studiengang}{Angewandte Informatik (BSc)} % Ihr Studiengang
\newcommand{\autor}{Dennis Pidun} % Ihr Name
\newcommand{\email}{pidund@uni-hildesheim.de}% Ihre aktuelle und g�ltige E-Mail-Adresse
\newcommand{\matnr}{296267} % Ihre Matrikelnummer

% Die Angaben zu uns:
\newcommand{\institut}{Institut f�r Informatik}
\newcommand{\arbeitsgruppe}{Arbeitsgruppe Software Systems Engineering}
\newcommand{\erstgutachter}{Prof. Dr. Klaus Schmid, SSE}% Ihr(e) ErstgutachterIn
\newcommand{\zweitgutachter}{Adam Krafczyk (MSc)}% Ihr(e) ZweitgutachterIn
\newcommand{\universitaet}{Universit�t Hildesheim\ \textbullet \ Universit�tsplatz 1 \ \textbullet \ D-31134 Hildesheim}
\newcommand{\adresse}{\arbeitsgruppe \ \textbullet \ \institut \\ \universitaet}

\newcommand{\version}{Version 0.0.1}% Die Version der Arbeit

% Wird 'projektarbeit' auf 'true' gesetzt, wird keine Eigenst�ndigkeitserkl�rung
% erzeugt.
\newboolean{projektarbeit}
\setboolean{projektarbeit}{true}

% Wird 'final' auf 'true' gesetzt, werden folgende �nderungen vorgenommen:
% -Entfernen von Datum in der Kopf- und Versionsnummer in der Fu�zeile
% -Entfernen von Datum und Versionsnummer vom Deckblatt
% -Es werden Leerseiten f�r den doppelseitigen Druck eingef�gt
\newboolean{final}
\setboolean{final}{false}





% Erstellung der Verzeichnisse und Glossars aktivieren
\makeindex
\makenomenclature
\makeglossaries 
\glstoctrue 


% Kopf- und Fu�zeilen, Seitenr�nder etc. anpassen
\input{Seitenstil}

% Eigene Definitionen f�r Silbentrennung laden
\include{Silbentrennung}

% Eigene LaTeX-Befehle laden
\include{Befehle}

% Hier beginnt das eigentliche Dokument
\begin{document}
\title{\titel}
\author{\autor}

% Setzt wie tief Abschnitte numeriert und ins Inhaltsverzeichnis
% aufgenommen werden sollen, hier: bis inkl. SubSubSection (z.B. 1.1.1.1).
\setcounter{secnumdepth}{3}
\setcounter{tocdepth}{3}

% Deckblatt einbinden
\include{Deckblatt}

% Setzen des Papierformats f�r den Rest des Dokuments
%\newgeometry{left=3.5cm, right=2.5cm, top=2.9cm, bottom=2.9cm}

% Bei finaler Fassung: Leere Seite als R�ckseite des Deckblatts 
\ifthenelse{\boolean{final}}{\cleardoublepage}{} 

% Selbstst�ndigkeitserkl�rung einbinden
\ifthenelse{\boolean{projektarbeit}}{}{\input{Erklaerung}}

% Bei finaler Fassung: Leere Seite als R�ckseite der Erkl�rung
\ifthenelse{\boolean{projektarbeit}}{}{\ifthenelse{\boolean{final}}{\cleardoublepage}{}}

% Abstract einbinden
% Vor dem Hauptteil werden die Seiten in gro�en r�mischen Ziffern nummeriert.
\pagenumbering{roman}

\section*{Kurzfassung}
\section*{Abstract}



% Bei finaler Fassung: Leere Seite als R�ckseite des Abstracts
\ifthenelse{\boolean{final}}{\cleardoublepage}{}

% Verzeichnisse drucken
\tableofcontents% Inhaltsverzeichnis

% Wenn Abbildungs- und Tabellenverzeichnis auf die selbe Seite sollen: 
\iftotalfigures\listoffigures\fi
\begingroup 
\let\clearpage\relax
\vspace{1cm} 
\iftotaltables\listoftables\fi
\endgroup
 
% Wenn Abbildungs- und Tabellenverzeichnis je auf einer eigenen Seite beginnen
% sollen: 
% \listoffigures % Abbildungsverzeichnis
% \listoftables % Tabellenverzeichnis

% Erstellen des Liste der Listings
\renewcommand{\lstlistlistingname}{Quellcode-Verzeichnis}
\iftotallstlistings\lstlistoflistings\fi

% F�r korrekte �berschrift in der Kopfzeile
\clearpage\markboth{\nomname}{\nomname}

% Drucken des Abk�rzungsverzeichnises
\printnomenclature
\label{cha:Abkuerzungsverzeichnis}
 
% Arabische Seitenzahlen im Hauptteil 
\ifthenelse{\boolean{final}}{\cleardoublepage}{\clearpage}
\pagenumbering{arabic}

% Die Inhaltskapitel aus "Inhalt.tex" einbinden
\begin{spacing}{\zeilenabstandHauptteil}
% Hier k�nnen die einzelnen Kapitel inkludiert werden. 
% Die Dateien m�ssen auf .tex enden. Diese Endung muss
% beim Inkludieren aber weggelassen werden.
% Info: \include und \input unterscheiden sich im wesentlichen darin, dass bei
% \include immer eine neue Seite angefangen wird.

\chapter{Einleitung}
\section{Motivation}
\section{Ziel dieser Arbeit}

\chapter{Background Information}
\section{Symbolic Execution}
Zeitgleich zur Ausf�hrung mit konkreten Werten, wird das zu testende Programm mithilfe von symbolischen Werten ausgef�hrt.
Um dies zu demonstrieren, stellt man sich eine Funktion f(x) und eine Funktion h(x, y) vor. Zur Vereinfachung ist die
Gr��e der verwendeten Funktionen gering. Zu betrachten sei folgender Code-Ausschnitt in Listing \ref{lst:function-fh}.

\begin{minipage}{\linewidth}
    \lstset{language=c, basicstyle=\footnotesize, showstringspaces=false,tabsize=2}
    \lstinputlisting[label={lst:function-fh},caption={Demonstrationsbeispiel von DART nach Godefroid}]{listings/function-fh.code}
\end{minipage}

Die Ausf�hrung eines Programms mithilfe von symbolischen Werten geschieht bei DART fortlaufend. Hierbei werden f�r jede
Verschachtelung Equivalente Gleichungen geschaffen, welche statt Programmvariablen symbolische Werte beinhalten. Somit
folgt \(x \mapsto x_0\) und \(y \mapsto y_0\). Mithilfe dieser symbolischen Werten lassen sich also symbolische
Formeln ableiten, welche aus der Verfolgung der Unterprogramme entstehen \cite{godefroid2005dart}. Hierzu werden fortlaufend
alle Verschachtelungen untersucht und wo es m�glich ist nachverfolgt. Beispielsweise sieht man in Listing \ref{lst:function-fh}
in Zeile 3 die Bedingung \(x \neq y\). Aus der Analyse zur symbolischen Ausf�hrung folgt daher \(x \neq y \mapsto x_0 \neq y_0\)
und f�r Zeile 4 \(f(x) = x+10 \mapsto 2*x_0 = x_0+10\). Die symbolischen Variablen \(x_0\) und \(y_0\) enthalten hier die
Speicheradressen f�r \(x\) und \(y\). Besondere Beachtung erh�lt hier der Ausdruck \(2*x_0\) da dieser �ber
\textit{dynamic tracing} aufgel�st wurde \cite{godefroid2005dart}.

\chapter{Static Test Data Generation}

\chapter{Dynamic Approach}
Beim dynamischen Ansatz von Korel wird das zu testende Programm zur Laufzeit analysiert und Eingabedaten generiert.
Zun�chst wird der Ansatz von Korel dadurch eingeschr�nkt, dass dieser  sich auf das Node Problem konzentriert.
Dieses wird definiert durch:
\begin{quote}
    Given node Y in a program. The goal is to find a program input x on which node Y will be executed \cite{korel1990dynamic}.
\end{quote}

Dieses Problem wurde in der Vergangenheit damit gel�st, dass dieses auf das Path Problem reduziert wurde \cite{korel1990dynamic}.
Bei diesem Ansatz werden nacheinander alle m�glichen Wege zu eine Node Y gew�hlt. F�r jeden m�glichen Weg werden dann
entsprechend Eingabewerte generiert, die zu Node Y f�hren k�nnten. Dies ben�tigt jedoch das L�sen einer Reihe von
Ungleichungen, die in \textit{if} -und \textit{repeat} Statements vorkommen \cite{ince1987automatic}. H�ufig wird daher
das Programm mit symbolischen Werten ausgef�hrt. Diese Ausf�hrung ist jedoch in bestimmten F�llen sehr aufwendig \cite{ince1987automatic},
weshalb hier andere Ans�tze verfolgt werden sollten. Zudem kann es h�ufiger dazu kommen, dass einige Wege nicht m�glich sind,
und daher Ressourcen f�r die Berechnung m�glicher Eingabewerte verschwendet werden \cite{korel1990dynamic}.

Korel schl�gt daher einen anderen Ansatz vor, bei dem die Selektierung des Weges wegf�llt \cite{korel1990dynamic}.
Hierzu wird das Programm mit einer zuf�lligen Eingabe ausgef�hrt. Bei jedem neuen Branch wird eine \textit{search procedure}
angesto�en, welche entscheidet, ob es zur Ausf�hrung des aktuellen Branches kommt \cite{korel1990dynamic}. Sofern die
aktuelle Eingabe nicht zur gew�nschten Node f�hrt, wird ein neuer Eingabewert generiert \cite{korel1990dynamic}.

//Vergleich zu \cite{korel1990automated}, was ist anders?

Damit Informationen �ber gewisse Branches zur Verf�gung stehen, m�ssen diese klassifiziert werden. Dazu unterteilt Korel
diese in 4 Kategorien:

\begin{enumerate}
    \item A critical branch.
    \item A required branch.
    \item A semi-critical branch.
    \item A non-essential branch.
\end{enumerate}

Somit braucht man die \textit{critical branches} nicht verwerten, da diese nicht zur gew�nschten Node Y f�hren. Interessanter
sind die \textit{required branches}, da diese n�her an die gew�nschte Node f�hren k�nnen \cite{korel1990dynamic}. Diese werden
also nicht verworfen. Au�erdem gibt es noch \textit{semi-critical branches}, welche Einfluss auf den Weg zur Node Y haben
und in Zyklen enden k�nnen. Diese werden f�r den Suchalgorithmus verworfen. Die \textit{non-essential branches} werden
nicht verworfen, da diese keinen Einfluss auf die Node Y haben k�nnen \cite{korel1990dynamic}.

\section{Generierung von neuen Eingabewerten}
Mit dem L�sen von sogenannten Subgoals beschreibt Korel die Generierung von Eingabewerten f�r das zu testende Programm.
Auf dem Weg von einer Start-Node zu einer Node Y, durchl�uft das Programm einen entsprechenden Pfad. Sollte das Programm
nun einen Weg einschlagen, wodurch es die Ziel-Node Y nicht mehr erreichen kann, wird versucht ein neuer Eingabewert
zu generieren. Dieser neue Eingabewert soll im Idealfall an dieser Beschr�nkung vorbeikommen, sodass es zur gew�nschten
Node Y gelangt \cite{korel1990dynamic}. Korel beschreibt dies als \textit{Solving Subgoals}. Hierbei wird versucht mithilfe
von neuen zuf�lligen Eingabewerten die Beschr�nkung aufzul�sen.

Um dies zu vereinfachen, geht Korel davon aus, dass es sich hierbei um einfache Gleichungen beziehungsweise Ungleichungen
handelt \cite{korel1990dynamic}. Somit ist \(E_1\) \fett{op} \(E_2\) die Form der Ausdr�cke, wobei \(E_1\) und \(E_2\)
simple arithmetische Ausdr�cke sind und \fett{op} ein Vergleichsoperator ist \cite{korel1990dynamic}. Diese Ausdr�cke
k�nnen nach Tabelle \ref{tab:bspTabelle} in folgedene Formen umgewandelt werden: \(F\) \fett{rel} 0.

\begin{table}[h]
    \centering
    \caption{Predicate zu Function Relation nach Korel \cite{korel1990dynamic}}
    \label{tab:bspTabelle}
    \begin{tabular}{|c|c|c|} \hline
    Branch Predicate & Branch Function \(F\) & \fett{rel} \\ \hline
    $E_1 > E_2$ & $E_2 - E_1$ & $<$ \\ \hline
    $E_1 \geq E_2$ & $E_2 - E_1$ & $\leq$ \\ \hline
    $E_1 < E_2$ & $E_1 - E_2$ & $<$ \\ \hline
    $E_1 \leq E_2$ & $E_1 - E_2$ & $\leq$ \\ \hline
    $E_1 = E_2$ & abs($E_1 - E_2$) & $=$ \\ \hline
    $E_1 \neq E_2$ & abs($E_1 - E_2$) & $\leq$ \\ \hline
    \end{tabular}
\end{table}



\section{Probleme}

......................................................................
DART attempts to cover all executable program paths \cite{godefroid2005dart}

Dynamic Approach: geht nur in eine Richtung, Fokus liegt auf das Finden zu Node Y

dynamic test generation does not deal with functions calls, unknown code segments (such as library functions),

how to check at run-time whether predictions about new test inputs are matched in the next run, and does
not discuss completeness. \cite{godefroid2005dart}
.......................................................................
\section{Test Data Generation for Distributed Software}

\chapter{Directed Automated Random Testing}
Das \textit{Directed Automated Random Testing} (DART) ist ein Werkzeug, welches in der Unit Testing Phase die Software-
Entwickler*innen unterst�tzen kann. Besonders f�r gr��ere Programme wird h�ufig eine Technik angewendet, bei der man
zuf�llige Eingabewerte f�r das zu testende Programm erzeugt. In einigen F�llen erreicht man dadurch, dass man entsprechende
Fehler im Quellcode finden kann. In anderen F�llen durchl�uft man dadurch jedoch nicht die gew�nschten Programmpfade,
wodurch sich vermeintliche Fehler nicht aufsp�ren lassen. An diesem Punkt kann DART unterst�tzend wirken, da es mithilfe
des Suchalgorithmus Eingabevektoren erzeugen kann, welche jeden Programmpfad ablaufen lassen k�nnen \cite{godefroid2005dart}.
Au�erdem kann DART Standardfehler erkennen. Unter anderem erkennt es Programmabst�rze, Assertion Violations und Non-Terminations
\cite{godefroid2005dart}. Da sich durch Randomtesting, keine hohe Codecoverage erhalten l�sst wird der Ansatz derart erweitert,
dass dieser zielgerichtet versucht gewisse Eingabewerte schnell zu erzeugen. Wenn beispielsweise eine Bedingung lautet:
``if (x == 10) then ...'', ist die Chance gering, dass man durch zuf�llige Eingaben die entsprechend richtige Eingabe
findet \cite{godefroid2005dart}. Um dies zu verhindern, verbindet DART \textit{automated interface extraction} mit
\textit{dynamic test generation} \cite{godefroid2005dart}. Dies sorgt daf�r, dass die Wahrscheinlichkeit, dass der
\textit{then-Bereich} der Bedingung ausgef�hrt wird, bei 0.5 liegt \cite{godefroid2005dart}. Somit ist die Zeit, die man
f�r �berfl�ssige Eingabewerte aufbringt deutlich geringer.

Um automatisiert Testdaten zu erzeugen, sind drei Techniken notwendig: (1) \textit{Auslesen der Interfaces},
(2) \textit{Generierung der Testdriver} und (3) \textit{Analyse des Programms} \cite{godefroid2005dart}. Damit DART
entsprechend Eingabevektoren bilden kann, wird das Programm statisch analysiert. Bei diesem Schritt werden alle externen
Interfaces ausgelesen und analysiert. Dies geschieht, sodass die externe Umgebung durch DART gesteuert werden kann.
In diesem Fall k�nnen Mockobjekte gebildet werden, welche es dann DART erlauben, die Ausf�hrung genauer zu steuern.
Aufbauend darauf kann dann die Generierung des Testdrivers begonnen werden. Dieser enth�lt alle n�tigen Schritte, welche
das Programm testen und Fehler aufdecken \cite{godefroid2005dart}. Im letzten Schritt wird das Programm nun analysiert. Hierzu wird untersucht,
wie sich das Programm verh�lt, wenn es zuf�llige Eingabewerte erh�lt \cite{godefroid2005dart}. Aufbauend darauf, werden
nun automatisch neue Testeingabewerte generiert, sodass die Ausf�hrung entlang aller Pfade gelenkt werden
kann \cite{godefroid2005dart}.

\section{Execution Model}
Durch die Art und Weise wie das \textit{Execution Model} aufgebaut ist, unterst�tzt es DART derart, dass es das Programm
sowohl konkret als auch symbolisch gleichzeitig ausf�hren kann. Mithilfe des Execution Models lassen sich neue Inputvektoren
bilden und das zu testende Programm analysieren \cite{godefroid2005dart}. Bei der Ausf�hrung mit symbolischen Werten
werden somit verschiedene Informationen �ber die Pfadbeschr�nkungen gesammelt \cite{godefroid2005dart}. Diese symbolischen
Werte m�ssen jedoch auch gespeichert und verarbeitet werden. Dies geschieht mithilfe von \textit{Symbolic Variables}.

Symbolic Variables sind Platzhalter f�r echte Variablen innerhalb eines Programms. F�r jeden Platzhalter erstellt DART
automatisiert ein Memorymapping \(M\), welches Zuweisungen von Speicheradressen zu 32-Bit Words beinhaltet \cite{godefroid2005dart}.
\begin{quote}
    We identify symbolic variables by their addresses. Thus in an expression, m denotes either a memory address or the
    symbolic variable identified by address m, depending on the context \cite{godefroid2005dart}.
\end{quote}
Somit kann man �ber die Adresse bestimmen, ob es sich hierbei um einen konkreten Wert oder einen symbolischen Wert
handelt. Dies hilft sp�ter bei der Analyse des Programms und kann bei der Generierung von Daten n�tzlich sein. Mithilfe
von Symbolic Variables kann man nun verschiedene Ausdr�cke ableiten \cite{godefroid2005dart}. Diese Ausdr�cke sollen
DART dabei helfen, das zu testende Programm besser zu verstehen. Unter anderem wird bez�glich \(e\)  zwischen \(m\)
(einfachen Speicheradressen), \(c\) (Konstanten), \(*(e, e')\) (Multiplikationen), \(\leq(e, e')\) (Vergleiche),
\(\lnot e'\) (Negation), \(*e'\) (Pointer) und weiteren Ausdr�cken unterschieden \cite{godefroid2005dart}. Somit werden
durch die symbolischen Ausdr�cke jegliche Ausdr�cke abgebildet, welche es in der Zielsprache ebenfalls gibt.

Ein weiterer wichtiger Punkt sind die \textit{Semantics of a Program}. Darunter f�llt beispielsweise der State eines
Programms. Godefroid et al. beschreibt dies als \textit{Transition System} \cite{godefroid2005dart}.
\begin{quote}
    A transistion system where a state represents the values of all variables and a program counter and transitions
    represent execution of a program statement resulting in change of state \cite{godefroid2005dart}.
\end{quote}
Jede Ausf�hrung des Programms ist also ein Durchlauf durch dieses System. Diesen Durchlauf beschreib Godefroid et al. als
\textit{Path} \cite{godefroid2005dart}. Da DART die Ausf�hrung des Programms sowohl mit konkreten als auch mit symbolischen
Werten startet, m�ssen die \textit{Semantics of a Program} auf Speicherebene definiert werden \cite{godefroid2005dart}.
Dies bedeutet, dass Statements �hnlich zu den Semantics gestaltet sind, diese jedoch lediglich simple Maschinenanweisungen
sind.

Damit DART Entscheidungen zu gewissen Statements treffen kann, werden zus�tzliche Informationen �ber die Statements
selber gespeichert. Wenn \(l\) eine Speicheradresse f�r ein Statement ist, welches nicht \textit{abort} oder \textit{halt}
ist, dann ist \(l + 1\) ebenfalls eine Adresse von einem Statement \cite{godefroid2005dart}. Dies bedeutet, dass es ein
\(l_0\) geben muss, welches die Adresse des Statements ist, welches den Start den Programms symbolisiert. Dies nennt
Godefroid et al. \textit{Statement Labels} \cite{godefroid2005dart}. Zur Vereinfachung unterscheiden Godefroid et al. dabei drei Arten
von Statements:
\begin{enumerate}
    \item conditional statement \(c\)
    \item assignment statement \(a\)
    \item \textit{abort} und \textit{halt}
\end{enumerate}
Somit gibt es M�glichkeiten zur Verzweigung, Zuweisung von Werten und das Beenden des Programms. Letzteres kann durch
einen Fehler verursacht werden oder durch ein normales Ende des Programms. Au�erdem wird eine weitere spezielle Funktion
definiert: \textit{statement - at(l, M)} \cite{godefroid2005dart}. Diese Funktion gibt also das n�chste Statement an.

Um die echten Werte, welche an einer bestimmte Stelle im Speicher enthalten sind zu erhalten, definieren Godefroid et al.
\textit{Concrete Semantics} \cite{godefroid2005dart}. Es wird hier eine Funktion \textit{evaluate - concrete(e, M)}
definiert. Diese Funktion evaluiert den Ausdruck \(e\) im Kontext von \(M\) und gibt anschlie�end ein 32-Bit Wert f�r
\(e\) zur�ck \cite{godefroid2005dart}.

Wie bereits gesagt, geht man davon aus, dass das Programm nur aus \textit{Conditional Statements}, \textit{Assignement Statements}
und \textit{Abort} oder \textit{Halt} besteht. Somit ist die \textit{Ausf�hrung} des Programms eine definierte Abfolge
von genau diesen Statements. Zur weiteren Vereinfachung gehen Godefroid et al. davon aus, dass es sich bei der Ausf�hrung
um eine alternierende Abfolge von \textit{Conditional Statements} und \textit{Assignement Statements} handelt, welche
immer mit einem \textit{Abort} oder \textit{Halt} endet \cite{godefroid2005dart}. Alternativ dazu k�nnte man sich
die Ausf�hrung als Baumstruktur vorstellen. Demnach haben \textit{Assignement Nodes} eine Folgenode und
\textit{Conditional Statements} demnach eine oder zwei Folgenodes. Die Endnodes, beziehungsweise Leaves, w�ren demnach
dann entweder \textit{Abort} oder \textit{Halt} \cite{godefroid2005dart}. Stellt man nun dem Programm \(P\) verschiedene
Eingabevektoren zur Verf�gung, f�hrt dies zu einer Sequenz von Ausf�hrungen entlang eines Pfades.

\section{Directed Search Algorithmus}
Der directed Search Algorithmus unterst�tzt DART bei der Findung von Eingabevektoren. �hnlich wie bei der dynamischen
Generierung von Testdaten wird das Programm zun�chst mit zuf�lligen Werten gestartet \cite{godefroid2005dart, korel1990dynamic}.
DART zielt hierbei darauf ab, Unterst�tzung f�r die zuf�llige Generierung von Testdaten zu bieten. Somit soll mithilfe
des Suchprozesses die Generierung in die richtige Richtung gelenkt werden k�nnen. Dabei sollen die verschiedenen Eingabevektoren
ein m�glichst gro�es Spektrum an verschiedenen Pfaden aufweisen k�nnen. F�r jede Ausf�hrung wird nun ein neuer Inputvektor
gebildert. Wie bereits erl�utert, ist der Inputvektor entscheidend f�r die Reihenfolge der Statements, die ausgef�hrt werden.
Dieser Inputvektor kann dann f�r die n�chsten Iterationen genutzt werden.

Die Instrumentalisierung des Programms beginnt also mit der Initialisierung des Inputvektors und des Stacks. Dieser
beruht auf zuf�lligen Daten. Anschlie�end wird f�r die \textit{Symbolic Execution} ein \textit{Symbolic Memory} angelegt.
Dieses wird immer dann mit neuen Informationen versorgt, wenn es ein Assignment Statement erkennt \cite{godefroid2005dart}.
Hier wird somit der Ausdruck \(e\) mithilfe des Memory Mapping und des Symbolic Memory symbolisch ausgewertet. Zeitgleich
wird f�r dieses Statement der konkrete Wert berechnet und im Memory Mapping gespeichert \cite{godefroid2005dart}. Da wir
wissen, dass nach einem Assignment Statement ein neues Statement folgen muss, wird der Zeiger f�r das Statement um eins
verschoben. Dieser Zeiger zeigt also nun auf das n�chste verf�gbare Statement. Im Falle eines Conditional Statements wird
der Ausdruck \(e\) sowohl konkret als auch symbolisch evaluiert \cite{godefroid2005dart}. Sollte die konkrete Evaluierung
den Wert \textit{wahr} zur�ckgeben, wird zum Speicher der \textit{Path Constraints} der symbolisch evaluierte Wert vom
Ausdruck \(e\) hinzugef�gt \cite{godefroid2005dart}. Au�erdem wird der Stack aktualisiert, sodass dieser �ber den Wert
der symbolischen Evaluierung Bescheid wei�. Im Falle, dass die konkrete Evaluierung den Wert \textit{falsch} zur�ck gibt,
wird zum Speicher der \textit{Path Constraints} der negierte symbolisch evaluierte Wert vom Ausdruck \(e\) hinzugef�gt
\cite{godefroid2005dart}. Der Stack wird nun mit der Information aktualisiert, dass der konkret evaluierte Wert \textit{falsch}
betrug \cite{godefroid2005dart}. Au�erdem wird das n�chste zu evaluierende Statement auf das n�chste verf�gbare Statement
gesetzt \cite{godefroid2005dart}. Sollte das Statement \textit{Abort} beinhalten, bricht DART an der Stelle ab und wirft
eine Exception, welche wiederum vom Testdriver aufgefangen werden kann. Im Falle eines \textit{Halt} Statements wird
der \textit{Path Constraint Solver} aufgerufen. Dies bedeutet, dass das Programm auf normalem Weg beendet wurde, ohne
dass ein Fehler gefunden wurde.

\subsection{Path Constraint Solver}
Der Path Constraint Solver sucht nach einem geeigneten Eingabewert, welcher f�r den gesamten Ausdruck \textit{wahr} als
R�ckgabewert liefert.


Als Fortsetzung dieses Kapitels soll also beschrieben werden, wie DART diese Pfadbeschr�nkungen aufl�st und schlussendlich
zu neuen Eingabewerten gelangt.




\section{Systematic Modular Automated Random Testing}
Ein Problem der Generierung von Testdaten bei DART ist die langsame Ausf�hrung f�r gr��ere Programme, sodass Godefroid
vorschl�gt dies kompositionell durchzuf�hren. So stellt Godefroid einen neuen Algorithmus vor, welchen sie
\textit{Systematic Modular Automated Random Testing} (SMART) nennen \cite{godefroid2007compositional}.

\chapter{Schlussbemerkungen}
\section{Fazit}
\section{Ausblick}


\end{spacing}

% Die Inhalte des Anhangs werden analog zu den Kapiteln eingebunden.
% Wenn kein Anhang vorhanden ist, m�ssen die n�chten 7 Zeilen auskommentiert
% werden (bis '\end{spaching}').
\begin{spacing}{\zeilenabstandAnhang}
\appendix
    \chapter{Anhang}
    \label{sec:Anhang}
    % Hier k�nnen die einzelnen Anh�nge inkludiert werden. 
% Die Dateien m�ssen auf .tex enden. Diese Endung muss
% beim inkludieren aber weggelassen werden.
% Info: \include und \input unterscheiden sich im wesentlichen darin, dass bei
% \include immer eine neue Seite angefangen wird.
%\input{Inhalt/Beispiele} % Vor finaler Abgabe entfernen!

\end{spacing}

% Glossar einbinden
% Ein Beispiel f�r einen Glossareintrag
% Nur Eintr�ge, die mindestens ein Mal referenziert wurden 
% (z.B.: \gls{computer}), tauchen im Glossar am Ende der Arbeit auf.
 \newglossaryentry{computer} {
   name=Computer,
   description={ist ein elektronisches Ger�t, das Daten verarbeitet}
 }
\label{sec:Glossar}

% Mit diesem Befehl werden nach nicht referenierte Glossareintr�ge
% ausgegeben (Nur zum Testen einkommentieren!)
% \glsaddall

% Glossar drucken
\printglossaries
% \addcontentsline{toc}{chapter}{Glossar}

% \ifthenelse{\boolean{final}}{\cleardoublepage}{\clearpage}
\clearpage
% \addcontentsline{toc}{chapter}{Bibliography}


% Literaturverzeichnis
%   Quelldatei ist: "Bibliographie.bib".
\bibliography{Bibliographie} % Aufruf: bibtex

% Verschiedene Presets f�r den Stil der Zitate und des Literaturverzeichnisses
% \bibliographystyle{plaindin} % Nur Zahlen, deutsch
% \bibliographystyle{plain} % Nur Zahlen, englisch
\bibliographystyle{alphadin} % Alphanumerische K�rzel, deutsch
% \bibliographystyle{alpha} % Alphanumerische K�rzel, englisch

\end{document}