% Meta-Informationen 
% Falls Umlaute oder ein *�* vorkommen:
\usepackage[latin1]{inputenc}

% Hier k�nnen Sie Informationen zur Arbeit, sich selbst und Ihren Betreuern
% hinterlegen.



\newcommand{\titel}{Search-Based Generation of Test Data - Ein praxisorientierter �berblick} % Name der Arbeit
% Test Data Generation Tools - the Search-Based Way
% Search-Based Test Data Generation Tools

\newcommand{\untertitel}{} % Optional mit Untertitil
% Art der Arbeit ggf. zus�tzlich der Titel der Veranstaltung
\newcommand{\art}{Seminararbeit}
\newcommand{\studiengang}{Angewandte Informatik (BSc)} % Ihr Studiengang
\newcommand{\autor}{Dennis Pidun} % Ihr Name
\newcommand{\email}{dennis.pidun@uni-hildesheim.de}% Ihre aktuelle und g�ltige E-Mail-Adresse
\newcommand{\matnr}{296267} % Ihre Matrikelnummer

% Die Angaben zu uns:
\newcommand{\institut}{Institut f�r Informatik}
\newcommand{\arbeitsgruppe}{Arbeitsgruppe Software Systems Engineering}
\newcommand{\erstgutachter}{Prof. Dr. Klaus Schmid, SSE}% Ihr(e) ErstgutachterIn
\newcommand{\zweitgutachter}{Adam Krafczyk (MSc)}% Ihr(e) ZweitgutachterIn
\newcommand{\universitaet}{Universit�t Hildesheim\ \textbullet \ Universit�tsplatz 1 \ \textbullet \ D-31134 Hildesheim}
\newcommand{\adresse}{\arbeitsgruppe \ \textbullet \ \institut \\ \universitaet}

\newcommand{\version}{Version 0.0.1}% Die Version der Arbeit

% Wird 'projektarbeit' auf 'true' gesetzt, wird keine Eigenst�ndigkeitserkl�rung
% erzeugt.
\newboolean{projektarbeit}
\setboolean{projektarbeit}{true}

% Wird 'final' auf 'true' gesetzt, werden folgende �nderungen vorgenommen:
% -Entfernen von Datum in der Kopf- und Versionsnummer in der Fu�zeile
% -Entfernen von Datum und Versionsnummer vom Deckblatt
% -Es werden Leerseiten f�r den doppelseitigen Druck eingef�gt
\newboolean{final}
\setboolean{final}{false}



