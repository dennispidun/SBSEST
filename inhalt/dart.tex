\chapter{Directed Automated Random Testing}
Das \textit{Directed Automated Random Testing} (DART) ist ein Werkzeug, welches in der Unit Testing Phase die Software-
Entwickler*innen unterst�tzen kann. Besonders f�r gr��ere Programme wird h�ufig eine Technik angewendet, bei der man
zuf�llige Eingabewerte f�r das zu testende Programm erzeugt. In einigen F�llen erreicht man dadurch, dass man entsprechende
Fehler im Quellcode finden kann. In anderen F�llen durchl�uft man dadurch jedoch nicht die gew�nschten Programmpfade,
wodurch sich vermeintliche Fehler nicht aufsp�ren lassen. An diesem Punkt kann DART unterst�tzend wirken, da es mithilfe
des Suchalgorithmus Eingabevektoren erzeugen kann, welche jeden Programmpfad ablaufen lassen k�nnen \cite{godefroid2005dart}.
Au�erdem kann DART Standardfehler erkennen. Unter anderem erkennt es Programmabst�rze, Assertion Violations und Non-Terminations
\cite{godefroid2005dart}. Da sich durch Randomtesting, keine hohe Codecoverage erhalten l�sst wird der Ansatz derart erweitert,
dass dieser zielgerichtet versucht gewisse Eingabewerte schnell zu erzeugen. Wenn beispielsweise eine Bedingung lautet:
``if (x == 10) then ...'', ist die Chance gering, dass man durch zuf�llige Eingaben die entsprechend richtige Eingabe
findet \cite{godefroid2005dart}. Um dies zu verhindern, verbindet DART \textit{automated interface extraction} mit
\textit{dynamic test generation} \cite{godefroid2005dart}. Dies sorgt daf�r, dass die Wahrscheinlichkeit, dass der
\textit{then-Bereich} der Bedingung ausgef�hrt wird, bei 0.5 liegt \cite{godefroid2005dart}. Somit ist die Zeit, die man
f�r �berfl�ssige Eingabewerte aufbringt deutlich geringer.

Um automatisiert Testdaten zu erzeugen, sind drei Techniken notwendig: (1) \textit{Auslesen der Interfaces},
(2) \textit{Generierung der Testdriver} und (3) \textit{Analyse des Programms} \cite{godefroid2005dart}. Damit DART
entsprechend Eingabevektoren bilden kann, wird das Programm statisch analysiert. Bei diesem Schritt werden alle externen
Interfaces ausgelesen und analysiert. Dies geschieht, sodass die externe Umgebung durch DART gesteuert werden kann.
In diesem Fall k�nnen Mockobjekte gebildet werden, welche es dann DART erlauben, die Ausf�hrung genauer zu steuern.
Aufbauend darauf kann dann die Generierung des Testdrivers begonnen werden. Dieser enth�lt alle n�tigen Schritte, welche
das Programm testen und Fehler aufdecken \cite{godefroid2005dart}. Im letzten Schritt wird das Programm nun analysiert. Hierzu wird untersucht,
wie sich das Programm verh�lt, wenn es zuf�llige Eingabewerte erh�lt \cite{godefroid2005dart}. Aufbauend darauf, werden
nun automatisch neue Testeingabewerte generiert, sodass die Ausf�hrung entlang aller Pfade gelenkt werden
kann \cite{godefroid2005dart}.

\section{Execution Model}
Durch die Art und Weise wie das \textit{Execution Model} aufgebaut ist, unterst�tzt es DART derart, dass es das Programm
sowohl konkret als auch symbolisch gleichzeitig ausf�hren kann. Mithilfe des Execution Models lassen sich neue Inputvektoren
bilden und das zu testende Programm analysieren \cite{godefroid2005dart}. Bei der Ausf�hrung mit symbolischen Werten
werden somit verschiedene Informationen �ber die Pfadbeschr�nkungen gesammelt \cite{godefroid2005dart}. Diese symbolischen
Werte m�ssen jedoch auch gespeichert und verarbeitet werden. Dies geschieht mithilfe von \textit{Symbolic Variables}.

Symbolic Variables sind Platzhalter f�r echte Variablen innerhalb eines Programms. F�r jeden Platzhalter erstellt DART
automatisiert ein Memorymapping \(M\), welches Zuweisungen von Speicheradressen zu 32-Bit Words beinhaltet \cite{godefroid2005dart}.
\begin{quote}
    We identify symbolic variables by their addresses. Thus in an expression, m denotes either a memory address or the
    symbolic variable identified by address m, depending on the context \cite{godefroid2005dart}.
\end{quote}
Somit kann man �ber die Adresse bestimmen, ob es sich hierbei um einen konkreten Wert oder einen symbolischen Wert
handelt. Dies hilft sp�ter bei der Analyse des Programms und kann bei der Generierung von Daten n�tzlich sein. Mithilfe
von Symbolic Variables kann man nun verschiedene Ausdr�cke ableiten \cite{godefroid2005dart}. Diese Ausdr�cke sollen
DART dabei helfen, das zu testende Programm besser zu verstehen. Unter anderem wird bez�glich \(e\)  zwischen \(m\)
(einfachen Speicheradressen), \(c\) (Konstanten), \(*(e, e')\) (Multiplikationen), \(\leq(e, e')\) (Vergleiche),
\(\lnot e'\) (Negation), \(*e'\) (Pointer) und weiteren Ausdr�cken unterschieden \cite{godefroid2005dart}. Somit werden
durch die symbolischen Ausdr�cke jegliche Ausdr�cke abgebildet, welche es in der Zielsprache ebenfalls gibt.

Ein weiterer wichtiger Punkt sind die \textit{Semantics of a Program}. Darunter f�llt beispielsweise der State eines
Programms. Godefroid beschreibt dies als \textit{Transition System} \cite{godefroid2005dart}.
\begin{quote}
    A transistion system where a state represents the values of all variables and a program counter and transitions
    represent execution of a program statement resulting in change of state \cite{godefroid2005dart}.
\end{quote}
Jede Ausf�hrung des Programms ist also ein Durchlauf durch dieses System. Diesen Durchlauf beschreib Godefroid als
\textit{Path} \cite{godefroid2005dart}. Da DART die Ausf�hrung des Programms sowohl mit konkreten als auch mit symbolischen
Werten startet, m�ssen die \textit{Semantics of a Program} auf Speicherebene definiert werden \cite{godefroid2005dart}.
Dies bedeutet, dass Statements �hnlich zu den Semantics gestaltet sind, diese jedoch lediglich simple Maschinenanweisungen
sind.

Damit DART Entscheidungen zu gewissen Statements treffen kann, werden zus�tzliche Informationen �ber die Statements
selber gespeichert. Wenn \(l\) eine Speicheradresse f�r ein Statement ist, welches nicht \textit{abort} oder \textit{halt}
ist, dann ist \(l + 1\) ebenfalls eine Adresse von einem Statement \cite{godefroid2005dart}. Dies bedeutet, dass es ein
\(l_0\) geben muss, welches die Adresse des Statements ist, welches den Start den Programms symbolisiert. Dies nennt
Godefroid \textit{Statement Labels} \cite{godefroid2005dart}. Zur Vereinfachung unterscheidet Godefroid dabei drei Arten
von Statements:
\begin{enumerate}
    \item conditional statement \(c\)
    \item assignment statement \(a\)
    \item \textit{abort} und \textit{halt}
\end{enumerate}
Somit gibt es M�glichkeiten zur Verzweigung, Zuweisung von Werten und das Beenden des Programms. Letzteres kann durch
einen Fehler verursacht werden oder durch ein normales Ende des Programms. Au�erdem wird eine weitere spezielle Funktion
definiert: \textit{statement - at(l, M)} \cite{godefroid2005dart}. Diese Funktion gibt also das n�chste Statement an.

Um die echten Werte, welche an einer bestimmte Stelle im Speicher enthalten sind zu erhalten, definiert Godefroid
\textit{Concrete Semantics} \cite{godefroid2005dart}. Es wird hier eine Funktion \textit{evaluate - concrete(e, M)}
definiert. Diese Funktion evaluiert den Ausdruck \(e\) im Kontext von \(M\) und gibt anschlie�end ein 32-Bit Wert f�r
\(e\) zur�ck \cite{godefroid2005dart}.

Wie bereits gesagt, geht man davon aus, dass das Programm nur aus \textit{Conditional Statements}, \textit{Assignement Statements}
und \textit{Abort} oder \textit{Halt} besteht. Somit ist die \textit{Ausf�hrung} des Programms eine definierte Abfolge
von genau diesen Statements. Zur weiteren Vereinfachung geht Godefroid davon aus, dass es sich bei der Ausf�hrung
um eine alternierende Abfolge von \textit{Conditional Statements} und \textit{Assignement Statements} handelt, welche
immer mit einem \textit{Abort} oder \textit{Halt} endet \cite{godefroid2005dart}. Alternativ dazu k�nnte man sich
die Ausf�hrung als Baumstruktur vorstellen. Demnach haben \textit{Assignement Nodes} eine Folgenode und
\textit{Conditional Statements} demnach eine oder zwei Folgenodes. Die Endnodes, beziehungsweise Leaves, w�ren demnach
dann entweder \textit{Abort} oder \textit{Halt} \cite{godefroid2005dart}. Stellt man nun dem Programm \(P\) verschiedene
Eingabevektoren zur Verf�gung, f�hrt dies zu einer Sequenz von Ausf�hrungen entlang eines Pfades.

\section{Directed Search Algorithmus}
Der directed Search Algorithmus unterst�tzt DART bei der Findung von Eingabevektoren. �hnlich wie bei der dynamischen
Generierung von Testdaten wird das Programm zun�chst mit zuf�lligen Werten gestartet \cite{godefroid2005dart, korel1990dynamic}.
DART zielt hierbei darauf ab, Unterst�tzung f�r die zuf�llige Generierung von Testdaten zu bieten. Somit soll mithilfe
des Suchprozess die Generierung in die richtige Richtung gelenkt werden k�nnen. Dabei sollen die verschiedenen Eingabevektoren
ein m�glichst gro�es Spektrum an verschiedenen Pfaden aufweisen k�nnen. F�r jede Ausf�hrung wird nun ein neuer Inputvektor
gebildert. Wie bereits erl�utert, ist der Inputvektor entscheidend f�r die Reihenfolge der Statements die ausgef�hrt werden.
Dieser Inputvektor kann dann f�r die n�chsten Iterationen genutzt werden.

Die Instrumentalisierung des Programms beginnt also mit der Initialisierung des Inputvektors und des Stacks. Dieser
beruht auf zuf�lligen Daten. Anschlie�end wird f�r die \textit{Symbolic Execution} ein \textit{Symbolic Memory} angelegt,
dieses wird immer dann mit neuen Informationen versorgt, wenn es ein Assignment Statement erkennt \cite{godefroid2005dart}.
Hier wird somit der Ausdruck \(e\) mithilfe des Memory Mapping und des Symbolic Memory symbolisch ausgewertet. Zeitgleich
wird f�r dieses Statement der konkrete Wert berechnet und im Memory Mapping gespeichert \cite{godefroid2005dart}. Da wir
wissen, dass nach einem Assignment Statement ein neues Statement folgen muss, wird der Zeiger f�r das Statement um eins
verschoben. Dieser Zeiger zeigt also nun auf das n�chste verf�gbare Statement. Im Falle eines Conditional Statements wird
der Ausdruck \(e\) sowohl konkret als auch symbolisch evaluiert \cite{godefroid2005dart}. Sollte die konkrete Evaluierung
den Wert \textit{wahr} zur�ckgeben, wird zum Speicher der \textit{Path Constraints} der symbolisch evaluierte Wert vom
Ausdruck \(e\) hinzugef�gt \cite{godefroid2005dart}. Au�erdem wird der Stack aktualisiert, sodass dieser �ber den Wert
der symbolischen Evaluierung Bescheid wei�. Im Falle, dass die konkrete Evaluierung den Wert \textit{falsch} zur�ck gibt,
wird zum Speicher der \textit{Path Constraints} der negierte symbolisch evaluierte Wert vom Ausdruck \(e\) hinzugef�gt
\cite{godefroid2005dart}. Der Stack wird nun mit der Information aktualisiert, dass der konkret evaluierte Wert \textit{falsch}
betrug \cite{godefroid2005dart}. Au�erdem wird das n�chste zu evaluierende Statement auf das n�chste verf�gbare Statement
gesetzt \cite{godefroid2005dart}. Sollte das Statement \textit{Abort} beinhalten, bricht DART an der Stelle ab und wirft
eine Exception, welche wiederum vom Testdriver aufgefangen werden kann. Im Falle eines \textit{Halt} Statements wird
der \textit{Path Constraint Solver} aufgerufen. Dies bedeutet, dass das Programm auf normalen Weg beendet wurde, ohne
dass ein Fehler gefunden wurde.

\subsection{Path Constraint Solver}
Der Path Constraint Solver sucht nach einem geeigneten Eingabewert, welcher f�r den gesamten Ausdruck \textit{wahr}
R�ckgabewert hat.




\section{Systematic Modular Automated Random Testing}
Ein Problem der Generierung von Testdaten bei DART ist die langsame Ausf�hrung f�r gr��ere Programme, sodass Godefroid et al.
vorschlagen dies kompositionell durchzuf�hren. So stellt Godefroid et al. einen neuen Algorithmus vor, welchen sie
\textit{Systematic Modular Automated Random Testing} (SMART) nennen \cite{godefroid2007compositional}.

Dieses Kapitel soll damit fortgesetzt werden, dass es zeigt, wie SMART generell funktioniert. Au�erdem werden Vor -und
Nachteile pr�sentiert und gezeigt welche Unterschiede zu DART selbst bestehen. Es werden also grundlegende Unterschiede
aufgezeigt und verdeutlicht warum genau das SMART besser macht als DART.
