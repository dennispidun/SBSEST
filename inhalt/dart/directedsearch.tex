\section{Directed Search Algorithmus}
Der directed Search Algorithmus unterst�tzt DART bei der Findung von Eingabevektoren. �hnlich wie bei der dynamischen
Generierung von Testdaten wird das Programm zun�chst mit zuf�lligen Werten gestartet \cite{godefroid2005dart, korel1990dynamic}.
DART zielt hierbei darauf ab, Unterst�tzung f�r die zuf�llige Generierung von Testdaten zu bieten. Somit soll mithilfe
des Suchprozesses die Generierung in die richtige Richtung gelenkt werden k�nnen. Dabei sollen die verschiedenen Eingabevektoren
ein m�glichst gro�es Spektrum an verschiedenen Pfaden aufweisen k�nnen. F�r jede Ausf�hrung wird nun ein neuer Inputvektor
gebildert. Wie bereits erl�utert, ist der Inputvektor entscheidend f�r die Reihenfolge der Statements, die ausgef�hrt werden.
Dieser Inputvektor kann dann f�r die n�chsten Iterationen genutzt werden.

Die Instrumentalisierung des Programms beginnt also mit der Initialisierung des Inputvektors und des Stacks. Dieser
beruht auf zuf�lligen Daten. Anschlie�end wird f�r die \textit{Symbolic Execution} ein \textit{Symbolic Memory} angelegt.
Dieses wird immer dann mit neuen Informationen versorgt, wenn es ein Assignment Statement erkennt \cite{godefroid2005dart}.
Hier wird somit der Ausdruck \(e\) mithilfe des Memory Mapping und des Symbolic Memory symbolisch ausgewertet. Zeitgleich
wird f�r dieses Statement der konkrete Wert berechnet und im Memory Mapping gespeichert \cite{godefroid2005dart}. Da wir
wissen, dass nach einem Assignment Statement ein neues Statement folgen muss, wird der Zeiger f�r das Statement um eins
verschoben. Dieser Zeiger zeigt also nun auf das n�chste verf�gbare Statement. Im Falle eines Conditional Statements wird
der Ausdruck \(e\) sowohl konkret als auch symbolisch evaluiert \cite{godefroid2005dart}. Sollte die konkrete Evaluierung
den Wert \textit{wahr} zur�ckgeben, wird zum Speicher der \textit{Path Constraints} der symbolisch evaluierte Wert vom
Ausdruck \(e\) hinzugef�gt \cite{godefroid2005dart}. Au�erdem wird der Stack aktualisiert, sodass dieser �ber den Wert
der symbolischen Evaluierung Bescheid wei�. Im Falle, dass die konkrete Evaluierung den Wert \textit{falsch} zur�ck gibt,
wird zum Speicher der \textit{Path Constraints} der negierte symbolisch evaluierte Wert vom Ausdruck \(e\) hinzugef�gt
\cite{godefroid2005dart}. Der Stack wird nun mit der Information aktualisiert, dass der konkret evaluierte Wert \textit{falsch}
betrug \cite{godefroid2005dart}. Au�erdem wird das n�chste zu evaluierende Statement auf das n�chste verf�gbare Statement
gesetzt \cite{godefroid2005dart}. Sollte das Statement \textit{Abort} beinhalten, bricht DART an der Stelle ab und wirft
eine Exception, welche wiederum vom Testdriver aufgefangen werden kann. Im Falle eines \textit{Halt} Statements wird
der \textit{Path Constraint Solver} aufgerufen. Dies bedeutet, dass das Programm auf normalem Weg beendet wurde, ohne
dass ein Fehler gefunden wurde.

\subsection{Path Constraint Solver}
Der Path Constraint Solver sucht nach einem geeigneten Eingabewert, welcher f�r den gesamten Ausdruck \textit{wahr} als
R�ckgabewert liefert.


Als Fortsetzung dieses Kapitels soll also beschrieben werden, wie DART diese Pfadbeschr�nkungen aufl�st und schlussendlich
zu neuen Eingabewerten gelangt.

