\section{Execution Model}
Durch die Art und Weise wie das \textit{Execution Model} aufgebaut ist, unterst�tzt es DART derart, dass es das Programm
sowohl konkret als auch symbolisch gleichzeitig ausf�hren kann. Mithilfe des Execution Models lassen sich neue Inputvektoren
bilden und das zu testende Programm analysieren \cite{godefroid2005dart}. Bei der Ausf�hrung mit symbolischen Werten
werden somit verschiedene Informationen �ber die Pfadbeschr�nkungen gesammelt \cite{godefroid2005dart}. Diese symbolischen
Werte m�ssen jedoch auch gespeichert und verarbeitet werden. Dies geschieht mithilfe von \textit{Symbolic Variables}.

Symbolic Variables sind Platzhalter f�r echte Variablen innerhalb eines Programms. F�r jeden Platzhalter erstellt DART
automatisiert ein Memorymapping \(M\), welches Zuweisungen von Speicheradressen zu 32-Bit Words beinhaltet \cite{godefroid2005dart}.
\begin{quote}
    We identify symbolic variables by their addresses. Thus in an expression, m denotes either a memory address or the
    symbolic variable identified by address m, depending on the context \cite{godefroid2005dart}.
\end{quote}
Somit kann man �ber die Adresse bestimmen, ob es sich hierbei um einen konkreten Wert oder einen symbolischen Wert
handelt. Dies hilft sp�ter bei der Analyse des Programms und kann bei der Generierung von Daten n�tzlich sein. Mithilfe
von Symbolic Variables kann man nun verschiedene Ausdr�cke ableiten \cite{godefroid2005dart}. Diese Ausdr�cke sollen
DART dabei helfen, das zu testende Programm besser zu verstehen. Unter anderem wird bez�glich \(e\)  zwischen \(m\)
(einfachen Speicheradressen), \(c\) (Konstanten), \(*(e, e')\) (Multiplikationen), \(\leq(e, e')\) (Vergleiche),
\(\lnot e'\) (Negation), \(*e'\) (Pointer) und weiteren Ausdr�cken unterschieden \cite{godefroid2005dart}. Somit werden
durch die symbolischen Ausdr�cke jegliche Ausdr�cke abgebildet, welche es in der Zielsprache ebenfalls gibt.

Ein weiterer wichtiger Punkt sind die \textit{Semantics of a Program}. Darunter f�llt beispielsweise der State eines
Programms. Godefroid et al. beschreibt dies als \textit{Transition System} \cite{godefroid2005dart}.
\begin{quote}
    A transistion system where a state represents the values of all variables and a program counter and transitions
    represent execution of a program statement resulting in change of state \cite{godefroid2005dart}.
\end{quote}
Jede Ausf�hrung des Programms ist also ein Durchlauf durch dieses System. Dieser Durchlauf l�uft laut Godefroid et al.
entlang eines \textit{Paths} \cite{godefroid2005dart}. Da DART die Ausf�hrung des Programms sowohl mit konkreten als auch
mit symbolischen Werten startet, m�ssen die \textit{Semantics of a Program} auf Speicherebene definiert werden \cite{godefroid2005dart}.
Dies bedeutet, dass Statements �hnlich zu den Semantics gestaltet sind, diese jedoch lediglich simple Maschinenanweisungen
sind.

Damit DART Entscheidungen zu gewissen Statements treffen kann, werden zus�tzliche Informationen �ber die Statements
selber gespeichert. Wenn \(l\) eine Speicheradresse f�r ein Statement ist, welches nicht \textit{abort} oder \textit{halt}
ist, dann ist \(l + 1\) ebenfalls eine Adresse von einem Statement \cite{godefroid2005dart}. Dies bedeutet, dass es ein
\(l_0\) geben muss, welches die Adresse des Statements ist, welches den Start des Programms symbolisiert. Dies nennt
Godefroid et al. \textit{Statement Labels} \cite{godefroid2005dart}. Zur Vereinfachung unterscheiden Godefroid et al. dabei drei Arten
von Statements:
\begin{enumerate}
    \item conditional statement \(c\)
    \item assignment statement \(a\)
    \item \textit{abort} und \textit{halt}
\end{enumerate}
Somit gibt es M�glichkeiten zur Verzweigung, Zuweisung von Werten und das Beenden des Programms. Letzteres kann durch
einen Fehler verursacht werden oder durch ein normales Ende des Programms. Au�erdem wird eine weitere spezielle Funktion
definiert: \textit{statement - at(l, M)} \cite{godefroid2005dart}. Diese Funktion gibt also das n�chste Statement an.

Um die echten Werte, welche an einer bestimmte Stelle im Speicher enthalten sind, zu erhalten, definieren Godefroid et al.
\textit{Concrete Semantics} \cite{godefroid2005dart}. Es wird hier eine Funktion \textit{evaluate - concrete(e, M)}
definiert. Diese Funktion evaluiert den Ausdruck \(e\) im Kontext von \(M\) und gibt anschlie�end ein 32-Bit Wert f�r
\(e\) zur�ck \cite{godefroid2005dart}.

Wie bereits gesagt, geht man davon aus, dass das Programm nur aus \textit{Conditional Statements}, \textit{Assignement Statements}
und \textit{Abort} oder \textit{Halt} besteht. Somit ist die \textit{Ausf�hrung} des Programms eine definierte Abfolge
von genau diesen Statements. Zur weiteren Vereinfachung gehen Godefroid et al. davon aus, dass es sich bei der Ausf�hrung
um eine alternierende Abfolge von \textit{Conditional Statements} und \textit{Assignement Statements} handelt, welche
immer mit einem \textit{Abort} oder \textit{Halt} endet \cite{godefroid2005dart}. Alternativ dazu k�nnte man sich
die Ausf�hrung als Baumstruktur vorstellen. Demnach haben \textit{Assignement Nodes} eine Folgenode und
\textit{Conditional Statements} demnach eine oder zwei Folgenodes. Die Endnodes, beziehungsweise Leaves, w�ren demnach
dann entweder \textit{Abort} oder \textit{Halt} \cite{godefroid2005dart}. Stellt man nun dem Programm \(P\) verschiedene
Eingabevektoren zur Verf�gung, f�hrt dies zu einer Sequenz von Ausf�hrungen entlang eines Pfades.
