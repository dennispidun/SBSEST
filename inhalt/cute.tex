\chapter{Concolic Unit Testing Engine for C}
Die \textit{Concolic Unit Testing Engine} (CUTE) ist ein weiteres Tool zur Generierung von Testdaten, welches auf
Suchverfahren aufbaut. Im Gegensatz zum Ansatz von Korel \cite{korel1990dynamic} und Godefroid et al. \cite{godefroid2005dart}
wird bei CUTE das Programm nicht mit zuf�lligen Eingabewerten ausgef�hrt. Diese zuf�llige Eingabe erzeugt n�mlich zwei
Probleme. Beispielsweise k�nnen viele verschiedene Eingaben zum gleichen Resultat f�hren, was die Ausf�hrungszeit
drastisch erh�ht \cite{sen2005cute}. Au�erdem ist die Wahrscheinlichkeit dar�ber Fehler zu finden sehr gering
\cite{sen2005cute}. Somit ben�tigt es einen Ansatz, welcher daf�r sorgt, dass diese Probleme verhindert werden und somit
die Ausf�hrungszeit verringert. Da es DART schwer f�llt Zeigeroperationen korrekt zu verarbeiten, f�hren Sen et al. somit
CUTE ein, welches ebenfalls in der Lage sein soll mit diesen Schwierigkeiten umzugehen.
\begin{quote}
    The key idea of our method is to represent inputs for the unit under test using a \textit{logical input map} that
    represents all inputs [...] and then to build constraints on these inputs by symbolically executing the code under
    test \cite{sen2005cute}.
\end{quote}
Somit basiert CUTE nicht auf der zuf�lligen Generierung von Testdaten. Es wird viel mehr versucht die Bedingungen symbolisch
zu l�sen, so dass hier keine Ausf�hrung des Programms selbst notwendig ist. Zur Aufl�sung der Bedingungen und zur
Berechnung von konkreten Werten wird die \textit{Logical Input Map} genutzt \cite{sen2005cute}. Um zu demonstrieren
wie CUTE arbeitet, sei zun�chst folgende Beispielfunktion aus Listing \ref{lst:testme} zu betrachten.

\begin{minipage}{\linewidth}
    \lstset{language=c, basicstyle=\footnotesize, showstringspaces=false,tabsize=2}
    \lstinputlisting[label={lst:testme},caption={testme Funktion nach Sen et al. \cite{sen2005cute}}]{listings/testme.code}
\end{minipage}

Wie man erkennen kann, ist hier ein Fehler eingebaut, welchen man durch bestimmte Eingabewerte erreichen kann. Demnach
h�ngt die Ausf�hrung von den Parametern \(*p\) und \(x\) ab \cite{sen2005cute}. CUTE arbeitet an dieser Stelle �hnlich
wie DART. Hierbei wird zun�chst der Parameter \(*p\) mit NULL aufgerufen. F�r \(x\) wird ein zuf�lliger Wert gebildet
\cite{godefroid2005dart, sen2005cute}. Beim Durchlaufen des Programms werden nun alle Pfadbeschr�nkungen vermerkt. Da der
Wert NULL f�r \(*p\) daf�r sorgt, dass die Auswertung der \(if\)-Anweisung \textit{falsch} als Ergebnis liefert, m�ssen
hier die symbolischen Ausdr�cke umgedreht werden \cite{sen2005cute}. Wenn $p_0$ und $x_0$ die symbolischen Variablen von
$*p$ und $x$ darstellen, dann kann man bei der Ausf�hrungen folgende Pfadbeschr�nkungen notieren: $x_0 > 0$ f�r die
erste \(if\)-Anweisung und $p_0 = NULL$ f�r den else-Zweig der zweiten \(if\)-Anweisung \cite{sen2005cute}.



Nachfolgend soll darauf eingegangen werden, wie CUTE generell funktioniert. Es werden Beispiele gezeigt, an denen der
Leser erkennen kann, wie CUTE in der Praxis eingesetzt werden kann. Au�erdem werden m�gliche Probleme aufgezeigt und
wie man diese mittels CUTE l�sen kann. Zudem wird auch hier gezeigt, wie ein Suchverfahren dazu beitragen kann, die
Qualit�t der Eingabewerte zu verbessern.
