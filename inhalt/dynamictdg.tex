\chapter{Dynamic Approach}
Beim dynamischen Ansatz von Korel wird das zu testende Programm zur Laufzeit analysiert und
Eingabedaten generiert. Zun�chst wird der Ansatz von Korel dadurch eingeschr�nkt, dass dieser
sich auf das Node Problem konzentriert. Dieses wird definiert durch:
\begin{quote}
    Given node Y in a program. The goal is to find a program input x on which node Y will be executed \cite{korel1990dynamic}.
\end{quote}

Dieses Problem wurde in der Vergangenheit damit gel�st, dass dieses auf das Path Problem reduziert wurde \cite{korel1990dynamic}.
Bei diesem Ansatz werden nacheinander alle m�glichen Wege zu eine Node Y gew�hlt. F�r jeden m�glichen Weg werden dann
entsprechend Eingabewerte generiert, die zu Node Y f�hren k�nnten. Dies ben�tigt jedoch das L�sen einer Reihe von
Ungleichungen, die in \textit{if} -und \textit{repeat} Statements vorkommen \cite{ince1987automatic}. H�ufig wird daher
das Programm mit symbolischen Werten ausgef�hrt. Diese Ausf�hrung ist jedoch in bestimmten F�llen sehr aufwendig \cite{ince1987automatic},
weshalb hier andere Ans�tze verfolgt werden sollten. Zudem kann es h�ufiger dazu kommen, dass einige Wege nicht m�glich sind,
und daher Ressourcen f�r die Berechnung m�glicher Eingabewerte verschwendet werden \cite{korel1990dynamic}.

Korel schl�gt daher einen anderen Ansatz vor, bei dem die Selektierung des Weges wegf�llt \cite{korel1990dynamic}.
Hierzu wird das Programm mit einer zuf�lligen Eingabe ausgef�hrt. Bei jedem neuen Branch wird eine \textit{search procedure}
angesto�en, welche entscheidet, ob es zur Ausf�hrung des aktuellen Branches kommt \cite{korel1990dynamic}. Sofern die
aktuelle Eingabe nicht zur gew�nschten Node f�hrt, wird ein neuer Eingabewert generiert \cite{korel1990dynamic}.

//Vergleich zu \cite{korel1990automated}, was ist anders?

Damit Informationen �ber gewisse Branches zur Verf�gung stehen, m�ssen diese klassifiziert werden. Dazu unterteilt Korel
diese in 4 Kategorien:

\begin{enumerate}
    \item A critical branch.
    \item A required branch.
    \item A semi-critical branch.
    \item A non-essential branch.
\end{enumerate}

Somit braucht man die \textit{critical branches} nicht verwerten, da diese nicht zur gew�nschten Node Y f�hren. Interessanter
sind die \textit{required branches}, da diese n�her an die gew�nschte Node f�hren k�nnen \cite{korel1990dynamic}. Diese werden
also nicht verworfen. Au�erdem gibt es noch \textit{semi-critical branches}, welche Einfluss auf den Weg zur Node Y haben
und in Zyklen enden k�nnen. Diese werden f�r den Suchalgorithmus verworfen. Die \textit{non-essential branches} werden
nicht verworfen, da diese keinen Einfluss auf die Node Y haben k�nnen \cite{korel1990dynamic}.

\section{Generierung von Eingabewerten}

\section{Probleme}

......................................................................
DART attempts to cover all executable program paths \cite{godefroid2005dart}

Dynamic Approach: geht nur in eine Richtung, Fokus liegt auf das Finden zu Node Y

dynamic test generation does not deal with functions calls, unknown code segments (such as library functions),

how to check at run-time whether predictions about new test inputs are matched in the next run, and does
not discuss completeness. \cite{godefroid2005dart}
.......................................................................
\section{Test Data Generation for Distributed Software}
