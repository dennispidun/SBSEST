\chapter{Background Information}
Mit dem Abschnitt \textit{Background Information} sollen Grundlagen geschaffen werden und der Leser soll in die Thematik
eingef�hrt werden. Dazu soll zun?chst \textit{Testing in General} erkl�rt werden. Anschlie�end werden verschiedene
\textit{Suchverfahren} vorgestellt werden. Im letzten Unterkapitel werden \textit{Symbolic Executions} erkl�rt und
aufgezeigt, welche Relevanz es f�r die verschiedenen Tools hat.

\section{Testverfahren}
Bei Testing in General werden verschiedene Methoden vorgestellt, wie man ein Programm testet, wie es in der Vergangenheit
h�ufig durchgef�hrt wurde und welche Potenziale automatisierte Tests haben k�nnen. Au�erdem werden Abgrenzungen
zu Methoden wie TDD gezeigt.
% Komponententests}
% Integegrationstests

\section{Heuristische Suchverfahren}
Hierbei sollen verschiedene (klassische) Suchverfahren -und Methoden vorgestellt werden, sodass der Leser ein grundlegendes
Verst�ndnis f�r Suchverfahren hat. Au�erdem wird aufgezeigt, wie man diese in Algorithmen integrieren kann.

\subsection{Hillclimb}

\section{Search-Based Software Engineering}
\subsection{Search-Based Software Testing}

