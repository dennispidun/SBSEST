\chapter{Background Information}
Mit dem Abschnitt \textit{Background Information} sollen Grundlagen geschaffen werden und der Leser soll in die Thematik
eingef�hrt werden. Dazu soll zun�chst \textit{Testing in General} erkl�rt werden. Anschlie�end werden verschiedene
\textit{Suchverfahren vorgestellt werden}. Im letzten Kapitel werden \textit{Symbolic Executions} erkl�rt werden und
aufgezeigt werden, welche Relevanz es f�r die verschiedenen Tools hat.

\section{Testing in General}
Bei Testing in General werden verschiedene Methoden vorgestellt, wie man ein Programm testet, wie es in der Vergangenheit
h�ufig durchgef�hrt wurde und (eventuell) welches Potenzial automatisierte Tests haben k�nnen. Au�erdem werden Abgrenzungen
zu Methoden wie TDD gezeigt.

\section{Suchverfahren}
Hierbei sollen verschiedene (klassische) Suchverfahren -und Methoden vorgestellt werden, sodass der Leser ein grundlegendes
Verst�ndnis f�r Suchverfahren hat. Au�erdem wird aufgezeigt, wie man diese in Algorithmen integrieren kann.

\section{Symbolic Execution}
Die Ausf�hrung eines Programms kann mittels konkreten Werten oder symbolischen Werten veranlasst werden. Zur Analyse des
zu testenden Programms wird es mithilfe von symbolischen Werten ausgef�hrt. Um zu demonstrieren wie dies funktioniert,
stelle man sich eine Funktion \(f(x)\) und eine Funktion \(h(x, y)\) vor. Zur Vereinfachung ist die Auspr�gung der verwendeten
Funktionen gering. Zu betrachten sei somit folgender Code-Ausschnitt in Listing \ref{lst:function-fh}.

\begin{minipage}{\linewidth}
    \lstset{language=c, basicstyle=\footnotesize, showstringspaces=false,tabsize=2}
    \lstinputlisting[label={lst:function-fh},caption={Demonstrationsbeispiel von DART nach Godefroid et al.}]{listings/function-fh.code}
\end{minipage}

Hierbei werden f�r jede Verschachtelung equivalente Gleichungen geschaffen, welche statt Programmvariablen symbolische
Werte beinhalten. Somit folgt \(x \mapsto x_0\) und \(y \mapsto y_0\). Mithilfe dieser symbolischen Werten lassen sich also
symbolische Formeln ableiten, welche aus der Verfolgung der Unterprogramme entstehen \cite{godefroid2005dart}. Hierzu
werden fortlaufend alle Verschachtelungen untersucht und wo es m�glich ist nachverfolgt. Beispielsweise sieht man in
Listing \ref{lst:function-fh} in Zeile 3 die Bedingung \(x \neq y\). Aus der Analyse zur symbolischen Ausf�hrung folgt
daher \(x \neq y \mapsto x_0 \neq y_0\) und f�r Zeile 4 \(f(x) = x+10 \mapsto 2*x_0 = x_0+10\). Die symbolischen
Variablen \(x_0\) und \(y_0\) enthalten hier die Speicheradressen f�r \(x\) und \(y\). Somit lassen sich �ber die
Ausf�hrung mithilfe von symbolischen Werten Gleichungssysteme bestimmen, welche dann gel�st werden k�nnen\cite{godefroid2005dart}.

