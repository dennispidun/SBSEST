\chapter{Background Information}
\section{Symbolic Execution}
Zur Analyse des zu testenden Programms wird es mithilfe von symbolischen Werten ausgef�hrt.
Um dies zu demonstrieren, stelle man sich eine Funktion f(x) und eine Funktion h(x, y) vor. Zur Vereinfachung ist die
Gr��e der verwendeten Funktionen gering. Zu betrachten sei somit folgender Code-Ausschnitt in Listing \ref{lst:function-fh}.

\begin{minipage}{\linewidth}
    \lstset{language=c, basicstyle=\footnotesize, showstringspaces=false,tabsize=2}
    \lstinputlisting[label={lst:function-fh},caption={Demonstrationsbeispiel von DART nach Godefroid}]{listings/function-fh.code}
\end{minipage}

Hierbei werden f�r jede Verschachtelung equivalente Gleichungen geschaffen, welche statt Programmvariablen symbolische
Werte beinhalten. Somit folgt \(x \mapsto x_0\) und \(y \mapsto y_0\). Mithilfe dieser symbolischen Werten lassen sich also
symbolische Formeln ableiten, welche aus der Verfolgung der Unterprogramme entstehen \cite{godefroid2005dart}. Hierzu
werden fortlaufend alle Verschachtelungen untersucht und wo es m�glich ist nachverfolgt. Beispielsweise sieht man in
Listing \ref{lst:function-fh} in Zeile 3 die Bedingung \(x \neq y\). Aus der Analyse zur symbolischen Ausf�hrung folgt
daher \(x \neq y \mapsto x_0 \neq y_0\) und f�r Zeile 4 \(f(x) = x+10 \mapsto 2*x_0 = x_0+10\). Die symbolischen
Variablen \(x_0\) und \(y_0\) enthalten hier die Speicheradressen f�r \(x\) und \(y\). Somit lassen sich �ber die
Ausf�hrung mithilfe von symbolischen Werten Gleichungssysteme bestimmen, welche dann gel�st werden k�nnen \cite{godefroid2005dart}.
